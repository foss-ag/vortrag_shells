\documentclass[12pt,utf8]{beamer}
\usepackage{graphicx}
\usepackage{xcolor}
\usepackage[ngerman]{babel}
\usepackage{wasysym}
\usepackage{Latex_Template/beamerthemeFOSSAG}

\title{Shells}
\subtitle{Alternativen zu Bash}
\author[N. Lenz]{Nicolas Lenz}
\institute[FOSS AG]{Free and Open Source Software AG\\Fakultät für Informatik}
\date{\today}

\begin{document}
	\titlepage
	
	\begin{frame}
	\frametitle{Einleitung}
	\begin{itemize}
		\item Bisher haben wir nur die bash als Shell benutzt
		\item Vielfältige Auswahl an Shells für verschiedene Einsatzbereiche
		\item Unterscheiden zwischen der Shell-Skriptsprache und dem interaktiven Modus
	\end{itemize}
	\end{frame}

	\begin{frame}
	\frametitle{Inhalt}
	\begin{itemize}
		\item Die Anfänge
		\begin{itemize}
			\item Thompson-Shell
			\item Bourne-Shell
			\item C-Shell
		\end{itemize}
		\pause
		\item Shell-Vielfalt
		\begin{itemize}
			\item Bourne-Again-Shell
			\item BusyBox
			\item (Debian-)Almquist-Shell
		\end{itemize}
		\pause
		\item Unsere Empfehlung
		\begin{itemize}
			\item Z-Shell
			\item Friendly Interactive Shell
		\end{itemize}
	\end{itemize}
	\end{frame}

	\begin{frame}
	\frametitle{Die Anfänge}
	\framesubtitle{Die Thompson-Shell}
	\begin{itemize}
		\item Erste UNIX-Shell, heute bekannt als \textbf{osh} für ``old shell``
		\item Verwendet in \textbf{UNIX} 1 bis 6 von 1971 bis 1975
		\item Führte Piping, einfache Kontrollstrukturen für Skripte und Wildcarding ein
	\end{itemize}
	\end{frame}

	\begin{frame}
	\frametitle{Die Anfänge}
	\framesubtitle{Die Bourne-Shell}
	\begin{itemize}
		\item Ursprünglich veröffentlicht \textbf{1977}
		\item Heute bekannt und immer noch genutzt als ``sh``
		\item 
	\end{itemize}
	\end{frame}

	\begin{frame}
	\frametitle{Die Anfänge}
	\framesubtitle{Die C-Shell}
	\end{frame}

	\begin{frame}
	\frametitle{Shell-Vielfalt}
	\framesubtitle{Die Bourne-Again-Shell}
	\end{frame}
	
	\begin{frame}
	\frametitle{Shell-Vielfalt}
	\framesubtitle{BusyBox}
	\end{frame}

	\begin{frame}
	\frametitle{Shell-Vielfalt}
	\framesubtitle{(Debian-)Almquist-Shell}
	\end{frame}

	\begin{frame}
	\frametitle{Unsere Empfehlung}
	\framesubtitle{Z-Shell}
	\end{frame}

	\begin{frame}
	\frametitle{Unsere Empfehlung}
	\framesubtitle{Friendly Interactive Shell}
	\end{frame}
\end{document}